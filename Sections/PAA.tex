\section{Propagation atmosphérique et antennes}
	Les ondes interragissent avec leur environnement, cela dépend de la fréquence, de la taille de l'objet rencontré, surface, matériaux, \dots
	
	Il existe différentes manières de propager les ondes :
	\begin{itemize}
		\item onde satellite
		\item onde ionosphère
		\item onde directe
		\item onde de sol
	\end{itemize}
	
	
	\subsection{Ondes directe}
		Les ondes sont envoyées en ligne droite à une autre antenne. La portée est limitée par l'horizon. Il y a aussi des interférences (distortion/dispersion) avec les ondes réfléchies sur le sol.
		
		\begin{figure}[H]
			\centering
			\includegraphics[width=0.7\textwidth]{img/onde_direct.png}
		\end{figure}
	
	\subsection{Ondes de sol}
		Ondes basse fréquences qui se propagent généralement le long du sol car le front des ondes basses fréquences se déplace perpendiculairement au sol.
		\begin{center}
			\begin{tabular}{|c|c|}
			\hline
			Fréquency & Range(Km)\\
			\hline
			100 kHz & 200\\
			1 MHz & 60\\
			10 MHz & 6\\
			100 MHz & 1.5\\
			\hline
		
		\end{tabular}
		\end{center}
		
	\subsection{Onde ionosphériques}
		Avec l'effet de l'ionisation de l'air par les UV du soleil et crée un "mur" contre les ondes et est réfléchissant (en partie).
		Il y a 4 couches précises de la plus proche a la plus éloignée, $D,E,F_1,F_2$. pendant la nuit il ne reste qu'une seule couche nommée $F$.
		
		\begin{figure}[H]
			\centering
			\includegraphics[width=0.7\textwidth]{img/refraction.png}
		\end{figure}
		
		Grâce à cela, on peut propager les ondes entre continents. Cela est fortement utilisé pour les ondes hautes fréquences.
		
		\textbf{MUF} est le Maximal Usable Frenquency, c'est la fréquence max pour laquelle on est certain à 50\% qu'elle est réfléchie par la ionosphère. Elle est différente en fonction du jour ou de la nuit.
		
	\subsection{Onde par satellite}
		On a un satellite géostationnaire, qui reste au même endroit par rapport à la terre, à qui on envoie le signal et qui le retransmet à d'autres antennes.
		
		\begin{figure}[H]
			\centering
			\includegraphics[width=0.7\textwidth]{img/satellite.png}
		\end{figure}
		
	\subsection{Antennes}
		C'est un dispositif qui peut émettre, ou recevoir des ondes électromagnétiques. Autour d'une antenne, on a a des champs magnétiques et électriques.
	
		\begin{figure}[H]
			\centering
			\includegraphics[width=0.35\textwidth]{img/champAntenne.png}
		\end{figure}
		
		\textbf{Onde TEM} (Transverse Electric-Magnetic), c'est un mode de propagation tel que les champs électriques et magnétiques sont tous les 2 orthogonaux à la direction de propagation.
		\begin{figure}[H]
			\centering
			\includegraphics[width=0.45\textwidth]{img/TEM.png}
		\end{figure}
		
		\textbf{Rayonnement} : Quand l'antenne envoie un rayonnement dans une direction, elle provoque aussi un rayonnement inverse non-désiré. Ce rayonnement a un angle d'ouverture de $\omega$. On peut en mesurer le gain : 
		\begin{equation}
			\text{Gain} = \cfrac{\text{Puissance dans la direction de puissance max}}{\text{Puissance dans cette direction si l'antenne était isotrope}}
		\end{equation}
		
		\begin{minipage}{.5\textwidth}
  \centering
  \includegraphics[width=.6\textwidth]{img/rayonnement.png}
\end{minipage}%
\begin{minipage}{.5\textwidth}
  \centering
  \includegraphics[width=.6\textwidth]{img/rayonnement2.png}
\end{minipage}

	Il arrive que les ondes envoyées par des antennes, rencontrent des objets/obstacles (couche ionosphérique, batiment...). Ces objets reflètent les ondes et entraînent un problème de trajet multiple. C'est un problème qui fait que le même signal arrive à des moments différents chez le récepteur ce qui entraine des dispersion et de la désynchronisation du signal. On sait qu'un signal qui varie vite a plus de chance d'avoir un problème de trajet multiple.
	
	Il existe différent type d'antennes :
	
	\subsubsection{Dipole $\lambda/2$}
	
		\begin{figure}[H]
			\centering
			\includegraphics[width=0.7\textwidth]{img/dipole.png}
		\end{figure}
		
		C'est une antenne avec 2 tiges de taille $\lambda /4$, les tensions au centre sont minimes et maximales en extrémité.
	
		\begin{minipage}{.6\textwidth}
  \centering
  \includegraphics[width=.5\textwidth]{img/dipole1.png}

\end{minipage}%
\begin{minipage}{.5\textwidth}
  \centering
  \includegraphics[width=.8\textwidth]{img/dipole2.png}

\end{minipage}
		On peut minimiser l'antenne avec le principe de miroir qui reflète et simule le signal de l'autre branche.
		
	\subsubsection{Endfire}
		2 dipoles de longueur $\lambda/2$ séparés par $\lambda/4$, l'antenne de droite alimente avec une avance de phase de 90° (donc $\lambda/4$) ce qui entraine en renforcement vers la gauche.
		
		\begin{figure}[H]
			\centering
			\includegraphics[width=0.6\textwidth]{img/endfire.png}
		\end{figure}
		
	\subsubsection{Yagi}
		Même principe que les antennes endfire. Composé d'un réflecteur, dipôle et d'un directeur. Antennes très précise car directivité très étroite. On peut augmenter la directivité en ajoutant des directeurs mais cela diminue la bande passante.
		
		\begin{figure}[H]
			\centering
			\includegraphics[width=0.35\textwidth]{img/yagi.png}
			\includegraphics[width=0.55\textwidth]{img/yagi2.png}
		\end{figure}
		
	\subsubsection{Parabolique}
	
		Utilise une parabole pour émettre/recevoir un signal en le point centre de la parabole.
		
		\begin{figure}[H]
			\centering
			\includegraphics[width=0.45\textwidth]{img/parabole.png}
			\includegraphics[width=0.45\textwidth]{img/parabole2.png}
		\end{figure}
	\subsubsection{Réseau d'antenne}
		Utilisé pour récupérer un signal faible en "agrégeant" des signaux similaire
		\begin{figure}[H]
			\centering
			\includegraphics[width=0.7\textwidth]{img/reseauAntennes.png}
		\end{figure}
